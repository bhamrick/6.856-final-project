
\documentclass[11pt]{article}

\usepackage[latin1]{inputenc}
\usepackage{amssymb}
\usepackage{amsmath}
\usepackage{amscd}
\usepackage{amsthm}
\usepackage{amsfonts}
\usepackage{enumerate}
\usepackage{graphicx}
\usepackage{url}
\usepackage[breaklinks=true,hyperref]{hyperref}
\usepackage{amssymb}
\usepackage[dvips]{color}
\usepackage{epsfig}
\usepackage{mathrsfs}
\usepackage{indentfirst}
\usepackage{subfig}

\include{header}

\newcommand{\SOE}{{\sf SO}(\exists)}
\newcommand{\FOL}{{\sf FO(LFP)}}

\begin{document}

\begin{center} \begin{LARGE} {\sc \bf Amplification with Operators on Complexity Classes} \vspace{6pt}

{\sc 18.405 Final Paper, Spring 2011} \vspace{9pt}

\end{LARGE} { \Large \textsc{Brian Hamrick and Travis Hance}}

\end{center}

\section{Introduction}



\pagebreak

\begin{thebibliography}{9}

\bibitem{Toda}Toda, S. \emph{PP is as Hard as the Polynomial-Time Hierarchy.} Siam Journal of Computing. Vol. 20, No. 5, pp. 865-877. Oct 1991.

\bibitem{Toda2} Toda, S.; Ogiwara, M.; , \emph{Counting classes are at least as hard as the polynomial-time hierarchy,} Structure in Complexity Theory Conference, 1991., Proceedings of the Sixth Annual, pp. 2-12, 30 Jun-3 Jul 1991.

\end{thebibliography}

\end{document}


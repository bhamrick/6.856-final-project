
\documentclass[11pt]{article}

\usepackage[latin1]{inputenc}
\usepackage{amssymb}
\usepackage{amsmath}
\usepackage{amscd}
\usepackage{amsthm}
\usepackage{amsfonts}
\usepackage{enumerate}
\usepackage{graphicx}
\usepackage{url}
\usepackage[breaklinks=true,hyperref]{hyperref}
\usepackage{amssymb}
\usepackage[dvips]{color}
\usepackage{epsfig}
\usepackage{mathrsfs}


\pdfpagewidth 8.5in
\pdfpageheight 11in
\topmargin -1in
\headheight 0in
\headsep 0in
\textheight 8.5in
\textwidth 6.5in
\oddsidemargin 0in
\evensidemargin 0in
\headheight 75pt
\headsep 0in
\footskip .75in


\newenvironment{ee}{\begin{enumerate}}{\end{enumerate}}
\newenvironment{ii}{\begin{itemize}}{\end{itemize}}


\newcommand{\argmax}{\arg\!\max}
\newcommand{\argmin}{\arg\!\min}

\newcommand{\Var}{\text{Var}}
\newcommand{\Cov}{\text{Cov}}
\renewcommand{\Pr}[2]{\text{Pr}_{#1} \left[ #2 \right]}

\def\RR{\mathbb R}
\def\CC{\mathbb C}
\def\QQ{\mathbb Q}
\def\ZZ{\mathbb Z}
\def\NN{\mathbb N}
\def\powset{\mathbb P}
\def\FF{\mathbb F}

\def\e{\epsilon}
\def\d{\delta}

\def\ds{\displaystyle}
\newcommand{\vs}[1]{\vspace{#1 pt}}

\def\tensor{\otimes}
\def\xor{\oplus}

\newcommand{\floor}[1]{\left\lfloor #1 \right\rfloor}
\newcommand{\ceil}[1]{\left\lceil #1 \right\rceil}
\newcommand{\field}[1]{\mathbb #1}
\newcommand{\inner}[2]{\langle #1,#2\, \rangle}
\newcommand{\norm}[2]{\| #1 \|_{#2}}
\newcommand{\ket}[1]{| #1 \rangle}
\newcommand{\bra}[1]{\langle #1 |}
\newcommand{\dirac}[2]{\langle #1 | #2\, \rangle}

\newcommand{\bracket}[1]{\langle #1 \rangle}
\newcommand{\paren}[1]{\left( #1 \right)}
\newcommand{\set}[1]{\left\{ #1 \right\}}

\newcommand{\bset}{\left\{0,1\right\}}

\newcommand{\inv}{^{-1}}
\newcommand{\til}{\widetilde}
\newcommand{\sign}{\mathrm{sgn}\;}
\renewcommand{\mod}{\text{ mod }}

\newcommand{\poly}{\text{poly}}
\newcommand{\polylog}{\text{polylog}}
\newcommand{\tsc}[1]{\textsc{#1}}

\newcommand{\Co}{{\sf Co-}}
\newcommand{\co}{{\sf co}}
\newcommand{\modpoly}{/ \text{poly}}
\newcommand{\SPACE}{{\sf SPACE}}
\newcommand{\TIME}{{\sf TIME}}
\def\D{{\sf D}}
\def\N{{\sf N}}
\def\P{{\sf P}}
\def\L{{\sf L}}
\def\E{{\sf E}}
\newcommand{\promise}{\textsf{promise}}

\newcommand{\NP}{{\sf NP}}
\newcommand{\PSPACE}{{\sf PSPACE}}
\newcommand{\EXP}{{\sf EXP}}

\newcommand{\BP}{{\sf BP}}

\newcommand{\NL}{{\sf NL}}

\newcommand{\NC}{{\sf NC}}
\newcommand{\AC}{{\sf AC}}
\newcommand{\RP}{{\sf RP}}
\newcommand{\BPP}{{\sf BPP}}
\newcommand{\PH}{{\sf PH}}
\newcommand{\PP}{{\sf PP}}
\newcommand{\IP}{{\sf IP}}
\newcommand{\AM}{{\sf AM}}
\newcommand{\MA}{{\sf MA}}
\newcommand{\PCP}{{\sf PCP}}

\newtheorem{theorem}{Theorem}[section]
\newtheorem{lemma}[theorem]{Lemma}
\newtheorem{proposition}[theorem]{Proposition}
\newtheorem{prop}[theorem]{Proposition}
\newtheorem{corollary}[theorem]{Corollary}
\newtheorem{conjecture}[theorem]{Conjecture}

%\theoremstyle{definition}
\newtheorem{example}[theorem]{Example}
\newtheorem{problem}[theorem]{Problem}
\newtheorem{definition}[theorem]{Definition}
\newtheorem{question}[theorem]{Question}
%\theoremstyle{remark}

\newtheorem{remark}[theorem]{Remark}
\numberwithin{equation}{section}

\renewcommand{\theequation}{\thesection.\arabic{equation}}



\newcommand{\SOE}{{\sf SO}(\exists)}
\newcommand{\FOL}{{\sf FO(LFP)}}

\begin{document}

\begin{center} \begin{LARGE} {\sc \bf Uniform Sampling from a Convex Region} \vspace{6pt}

{\sc 6.856 Final Paper, Spring 2011} \vspace{9pt}

\end{LARGE} { \Large \textsc{Brian Hamrick and Travis Hance}}

\end{center}

\section{Introduction}

~~~ We consider the problem of randomly and uniformly sampling a point from a convex region in $n$-dimensional space. This has applications, for instance, in convex optimization \cite{Dabbene}. We phrase the problem as follows: we are given a convex region $\mathcal{R}$ in the form of a \emph{membership oracle}, a black box which can determine whether any point $x$ is in $\mathcal{R}$, along with a starting point $x_0$ in $\mathcal{R}$. The task is to sample a point from the region with with a distribution as close to uniform as possible.

Here, we study two Markov chain-based methods for sampling, the \emph{ball-walking} method and the \emph{hit-and-run} method, and we see how they perform comparitively in practice. The ball-walking method is the simpler of the two. We walk around the region in the following way: from a point $x$, we choose a random point in a ball around that point. If this new point lies in the convex region, then we jump to it; otherwise, we stay at $x$. We get a Markov chain $x_0, x_1, ...$, and after sufficiently many iterations, we take the point we are at to be our random point.

The hit-and-run method also performs a random walk, but in a different way. At a point $x$, we first choose a random direction, uniformly. Now consider the two-sided line $\ell$ which goes through $x$ which goes in this direction. Choose a point uniformly randomly from $\ell\cap\mathcal{R}$, and jump to that point.

It is known that both of these transition functions have a uniform stationary distribution, and that the distributions of the Markov chain eventually converge to these uniform distributions. In theory, the hit-and-run method converges must faster: the ball-walking method requires at least an amount of time exponential in the number of dimensions $n$ to get an approximation to the uniform distribution that is within the desired precision, whereas the hit-and-run method requires only polynomial time. However, in this paper we introduce \emph{adaptive ball-walking}, a modification of ball-walking which adjusts the size of the ball in hopes of reducing the amount of time necessary.

The aim of this paper is to compare the performance in practice of these three methods, which we have implemented and tested on high-dimensional convex regions.

\section{Ball Walking}

\section{Adaptive Ball Walking}

\section{Hit and Run}

\section{Implementation Details}

\section{Results}

\section{Conclusion}

\pagebreak

\begin{thebibliography}{99}

\bibitem{Dabbene} Dabbene, F., ``A randomized cutting plane scheme for convex optimization,'' Computer-Aided Control Systems, 2008. CACSD 2008. IEEE International Conference on , vol., no., pp.120-125, 3-5 Sept. 2008.

\bibitem{Vempala} Lov\'asz, L. and Vempala, S. Hit-and-Run from a corner. \emph{SIAM Journal on Computing}, 35(4):9851005, 2006.

\bibitem{Metropolis} Metropolis, N., Rosenbluth, M.N., Teller, A.H., Teller, E. ``Equations of State Calculations by Fast Computing Machines.'' \emph{Journal of Chemical Physics} \textbf{21} (6) pp. 1087-1092, June, 1953.

\bibitem{Smith} Smith, R.L. \emph{Efficient Monte-Carlo procedures for generating points uniformly distributed over
bounded regions}, Oper. Res., 32 (1984), pp. 1296-1308.

\end{thebibliography}

\end{document}


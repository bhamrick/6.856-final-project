
\documentclass[11pt]{article}

\usepackage[latin1]{inputenc}
\usepackage{amssymb}
\usepackage{amsmath}
\usepackage{amscd}
\usepackage{amsthm}
\usepackage{amsfonts}
\usepackage{enumerate}
\usepackage{graphicx}
\usepackage{url}
\usepackage[breaklinks=true,hyperref]{hyperref}
\usepackage{amssymb}
\usepackage[dvips]{color}
\usepackage{epsfig}
\usepackage{mathrsfs}
\usepackage{indentfirst}


\pdfpagewidth 8.5in
\pdfpageheight 11in
\topmargin -1in
\headheight 0in
\headsep 0in
\textheight 8.5in
\textwidth 6.5in
\oddsidemargin 0in
\evensidemargin 0in
\headheight 75pt
\headsep 0in
\footskip .75in


\newenvironment{ee}{\begin{enumerate}}{\end{enumerate}}
\newenvironment{ii}{\begin{itemize}}{\end{itemize}}


\newcommand{\argmax}{\arg\!\max}
\newcommand{\argmin}{\arg\!\min}

\newcommand{\Var}{\text{Var}}
\newcommand{\Cov}{\text{Cov}}
\renewcommand{\Pr}[2]{\text{Pr}_{#1} \left[ #2 \right]}

\def\RR{\mathbb R}
\def\CC{\mathbb C}
\def\QQ{\mathbb Q}
\def\ZZ{\mathbb Z}
\def\NN{\mathbb N}
\def\powset{\mathbb P}
\def\FF{\mathbb F}

\def\e{\epsilon}
\def\d{\delta}

\def\ds{\displaystyle}
\newcommand{\vs}[1]{\vspace{#1 pt}}

\def\tensor{\otimes}
\def\xor{\oplus}

\newcommand{\floor}[1]{\left\lfloor #1 \right\rfloor}
\newcommand{\ceil}[1]{\left\lceil #1 \right\rceil}
\newcommand{\field}[1]{\mathbb #1}
\newcommand{\inner}[2]{\langle #1,#2\, \rangle}
\newcommand{\norm}[2]{\| #1 \|_{#2}}
\newcommand{\ket}[1]{| #1 \rangle}
\newcommand{\bra}[1]{\langle #1 |}
\newcommand{\dirac}[2]{\langle #1 | #2\, \rangle}

\newcommand{\bracket}[1]{\langle #1 \rangle}
\newcommand{\paren}[1]{\left( #1 \right)}
\newcommand{\set}[1]{\left\{ #1 \right\}}

\newcommand{\bset}{\left\{0,1\right\}}

\newcommand{\inv}{^{-1}}
\newcommand{\til}{\widetilde}
\newcommand{\sign}{\mathrm{sgn}\;}
\renewcommand{\mod}{\text{ mod }}

\newcommand{\poly}{\text{poly}}
\newcommand{\polylog}{\text{polylog}}
\newcommand{\tsc}[1]{\textsc{#1}}

\newcommand{\Co}{{\sf Co-}}
\newcommand{\co}{{\sf co}}
\newcommand{\modpoly}{/ \text{poly}}
\newcommand{\SPACE}{{\sf SPACE}}
\newcommand{\TIME}{{\sf TIME}}
\def\D{{\sf D}}
\def\N{{\sf N}}
\def\P{{\sf P}}
\def\L{{\sf L}}
\def\E{{\sf E}}
\newcommand{\promise}{\textsf{promise}}

\newcommand{\NP}{{\sf NP}}
\newcommand{\PSPACE}{{\sf PSPACE}}
\newcommand{\EXP}{{\sf EXP}}

\newcommand{\BP}{{\sf BP}}

\newcommand{\NL}{{\sf NL}}

\newcommand{\NC}{{\sf NC}}
\newcommand{\AC}{{\sf AC}}
\newcommand{\RP}{{\sf RP}}
\newcommand{\BPP}{{\sf BPP}}
\newcommand{\PH}{{\sf PH}}
\newcommand{\PP}{{\sf PP}}
\newcommand{\IP}{{\sf IP}}
\newcommand{\AM}{{\sf AM}}
\newcommand{\MA}{{\sf MA}}
\newcommand{\PCP}{{\sf PCP}}

\newtheorem{theorem}{Theorem}[section]
\newtheorem{lemma}[theorem]{Lemma}
\newtheorem{proposition}[theorem]{Proposition}
\newtheorem{prop}[theorem]{Proposition}
\newtheorem{corollary}[theorem]{Corollary}
\newtheorem{conjecture}[theorem]{Conjecture}

%\theoremstyle{definition}
\newtheorem{example}[theorem]{Example}
\newtheorem{problem}[theorem]{Problem}
\newtheorem{definition}[theorem]{Definition}
\newtheorem{question}[theorem]{Question}
%\theoremstyle{remark}

\newtheorem{remark}[theorem]{Remark}
\numberwithin{equation}{section}

\renewcommand{\theequation}{\thesection.\arabic{equation}}



\newcommand{\SOE}{{\sf SO}(\exists)}
\newcommand{\FOL}{{\sf FO(LFP)}}

\begin{document}

\begin{center} \begin{LARGE} {\sc \bf Uniform Sampling from a Convex Region} \vspace{6pt}

{\sc 6.856 Final Paper, Spring 2011} \vspace{9pt}

\end{LARGE} { \Large \textsc{Brian Hamrick and Travis Hance}}

\end{center}

\section{Introduction}

We consider the problem of randomly and uniformly sampling a point from a convex region in $n$-dimensional space. This has applications, for instance, in convex optimization \cite{Dabbene}. We phrase the problem as follows: we are given a convex region $\mathcal{R}$ in the form of a \emph{membership oracle}, a black box which can determine whether any point $x$ is in $\mathcal{R}$, along with a starting point $x_0$ in $\mathcal{R}$. The task is to sample a point from the region with with a distribution as close to uniform as possible.

Here, we study two Markov chain-based methods for sampling, the \emph{Metropolis} method \cite{Metropolis}, and the \emph{hit-and-run} \cite{Vempala} method, and we see how they perform comparitively in practice. The etropolis method is the simpler of the two. We walk around the region in the following way: from a point $x$, we choose another random point in a ball around that point from some distribution. If this new point lies in the convex region, then we jump to it; otherwise, we stay at $x$. We get a Markov chain $x_0, x_1, ...$, and after sufficiently many iterations, we take the point we are at to be our random point.

In this paper, we study two particular cases of the Metropolis method. The first method is the \emph{ball-walking} method. When at a given point, this method chooses a random point in a ball of some fixed radius centered at that point. If that point is in $\mathcal{R}$, then it jumps to that point. The other method is the \emph{gaussian-walking} method, which chooses its point instead by selecting each coordinate from a gaussian centered at the current point.

The hit-and-run method is another random walk, but it has the advantage that it is guaranteed (almost surely) to move to another point. At a point $x$, we first choose a random direction, uniformly. Now consider the two-sided line $\ell$ which goes through $x$ which goes in this direction. Choose a point uniformly randomly from $\ell\cap\mathcal{R}$, and jump to that point.

It is known that all of these transition functions have a uniform stationary distribution, and that the distributions of the Markov chain eventually converge to these uniform distributions. In theory, the hit-and-run method converges must faster: the metropolis methods require in the worst case at least an amount of time exponential in the number of dimensions $n$ to get an approximation to the uniform distribution that is within the desired precision, whereas the hit-and-run method requires only polynomial time regardless of the starting distribution. In this paper we introduce \emph{adaptive metropolis}, modifications which adjust the transition distribution in hopes of reducing the required time in practice, although we are not aware of any theoretical improvements.

The aim of this paper is to compare the performance in practice of these methods, which we have implemented and tested on high-dimensional convex regions.

\section{The Metropolis Method}

The Metropolis methods both have the same basic structure. A transition in the Markov chain works simply by choosing another point from some distribution centered at our current location; if that point is in the convex region $\mathcal{R}$, then we move to that point; otherwise, we stay. The only difference between the two methods is the distribution we use. Ball-walking chooses a point randomly from a ball, and Metropolis chooses each coordinate from a Gaussian.

First, we want to show that these two transition functions both have the uniform distribution as a stationary distribution. To show this, we use the following fact, from \cite{Smith}.

\begin{theorem} \end{theorem}

It is also shown in \cite{Smith} that the distributions from a Markov chain eventually converge to this stationary uniform distribution. However, in the next section, we will see that this can take a very long time to converge, and we modify the algorithm in an attempt to fix this problem.

%we can also show that these algorithms may take exponentially long to converge. Suppose $\mathcal{R}$ is an $n$-dimensional cube $[-1,1]^n$, and suppose $x_0 = (1 - \epsilon, 1-\epsilon, ..., 1-\epsilon)$ for some small $\epsilon$. If we sample a point randomly from our transition distribution, then the probability that we choose a point in $\mathcal{R}$ is exponentially small in $n$. Thus, we end up ``stuck" at this point for a very long time. We attempt to fix this problem in the next section.

\section{Adaptive Metropolis}

Metropolis Markov chain methods, which include the basic ball walking method, have two major factors which can cause increased runtime. The first and more theoretically important is that if the current point is close to a corner of the region, then the chance of a step succeeding can be exponentially small. This arises even in simple cases such as a cube or a simplex if the sampling algorithm is given a point near a corner as the initial point. It also can occur if the step size is too large, the extreme case of which is sampling from a ball that is much larger than the entire admissible region. 

In this case shrinking the step size would clearly help, but it is not difficult to see that a small step size also mitigates the effect of nearby corners, as a larger fraction of the surrounding ball lies inside the admissible region.

The other contribution to the length of time required for Metropolis methods is that if the step size is too small, it takes a long time to walk from one point in the admissible region to another. As it is certainly not possible for the Markov chain to mix before it is able to reach any other given point, a small step size causes the state space to have a large diameter, and thus a long mixing time.

In order to find a good balance between the two main contributors to runtime stated above, we introduce an adaptive walk, which we run for some time before running the standard ball walking or Metropolis method. This adaptive walk uses the success or failure of a step to guess whether its current step size is too small or too large, and then compensates by modifying it by a constant factor. 

In the case of ball walking, our step size is the radius of the ball around the current point from which we draw the next point, and in the Metropolis method the step size corresponds to the standard deviation of the normal distributions used. However, it is not clear that this procedure has the uniform distribution as its stationary distribution, so we follow this adaptive procedure with the nonadaptive version, using the step size left by the adaptive method.

Since the nonadaptive methods converge to the stationary distribution from any starting distribution, this yields a method that converges to the stationary distribution, but also attempts to find a good step size in order to converge more rapidly.

\section{Hit and Run}

\section{Implementation Details}

\section{Results}

\section{Conclusion}

\pagebreak

\begin{thebibliography}{99}

\bibitem{Dabbene} Dabbene, F., ``A randomized cutting plane scheme for convex optimization,'' Computer-Aided Control Systems, 2008. CACSD 2008. IEEE International Conference on , vol., no., pp.120-125, 3-5 Sept. 2008.

\bibitem{Vempala} Lov\'asz, L. and Vempala, S. Hit-and-Run from a corner. \emph{SIAM Journal on Computing}, 35(4):9851005, 2006.

\bibitem{Metropolis} Metropolis, N., Rosenbluth, M.N., Teller, A.H., Teller, E. ``Equations of State Calculations by Fast Computing Machines.'' \emph{Journal of Chemical Physics} \textbf{21} (6) pp. 1087-1092, June, 1953.

\bibitem{Smith} Smith, R.L. \emph{Efficient Monte-Carlo procedures for generating points uniformly distributed over
bounded regions}, Oper. Res., 32 (1984), pp. 1296-1308.

\end{thebibliography}

\end{document}


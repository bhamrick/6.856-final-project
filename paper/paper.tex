
\documentclass[11pt]{article}

\usepackage[latin1]{inputenc}
\usepackage{amssymb}
\usepackage{amsmath}
\usepackage{amscd}
\usepackage{amsthm}
\usepackage{amsfonts}
\usepackage{enumerate}
\usepackage{graphicx}
\usepackage{url}
\usepackage[breaklinks=true,hyperref]{hyperref}
\usepackage{amssymb}
\usepackage[dvips]{color}
\usepackage{epsfig}
\usepackage{mathrsfs}

\include{header}

\newcommand{\SOE}{{\sf SO}(\exists)}
\newcommand{\FOL}{{\sf FO(LFP)}}

\begin{document}

\begin{center} \begin{LARGE} {\sc \bf Uniform Sampling from a Convex Region} \vspace{6pt}

{\sc 6.856 Final Paper, Spring 2011} \vspace{9pt}

\end{LARGE} { \Large \textsc{Brian Hamrick and Travis Hance}}

\end{center}

\section{Introduction}

We consider the problem of randomly and uniformly sampling a point from a convex region in $n$-dimensional space. This has applications, for instance, in convex optimization \cite{Dabbene}. We phrase the problem as follows: we are given a convex region $\mathcal{R}$ in the form of a \emph{membership oracle}, a black box which can determine whether any point $x$ is in $\mathcal{R}$, along with a starting point $x_0$ in $\mathcal{R}$. The task is to sample a point from the region with with a distribution as close to uniform as possible.

Here, we study two Markov chain-based methods for sampling, the ball-walking method and the hit-and-run method and see how they perform comparitively in practice.

\section{Ball Walking}

\section{Adaptive Ball Walking}

\section{Hit and Run}

\section{Implementation Details}

\section{Results}

\section{Conclusion}

\pagebreak

\begin{thebibliography}{99}

\bibitem{Dabbene} Dabbene, F., ``A randomized cutting plane scheme for convex optimization," Computer-Aided Control Systems, 2008. CACSD 2008. IEEE International Conference on , vol., no., pp.120-125, 3-5 Sept. 2008.

\bibitem{Vempala} L. Lov\'asz and S. Vempala. Hit-and-Run from a corner. \emph{SIAM Journal on Computing}, 35(4):9851005, 2006.

\bibitem{Smith} R.L. Smith, \emph{Efficient Monte-Carlo procedures for generating points uniformly distributed over
bounded regions}, Oper. Res., 32 (1984), pp. 1296-1308.

\end{thebibliography}

\end{document}


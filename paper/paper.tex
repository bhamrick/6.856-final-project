
\documentclass[11pt]{article}

\usepackage[latin1]{inputenc}
\usepackage{amssymb}
\usepackage{amsmath}
\usepackage{amscd}
\usepackage{amsthm}
\usepackage{amsfonts}
\usepackage{enumerate}
\usepackage{graphicx}
\usepackage{url}
\usepackage[breaklinks=true,hyperref]{hyperref}
\usepackage{amssymb}
\usepackage[dvips]{color}
\usepackage{epsfig}
\usepackage{mathrsfs}

\include{header}

\newcommand{\SOE}{{\sf SO}(\exists)}
\newcommand{\FOL}{{\sf FO(LFP)}}

\begin{document}

\begin{center} \begin{LARGE} {\sc \bf Uniform Sampling from a Convex Region} \vspace{6pt}

{\sc 6.856 Final Paper, Spring 2011} \vspace{9pt}

\end{LARGE} { \Large \textsc{Brian Hamrick and Travis Hance}}

\end{center}

\section{Introduction}

\section{Ball Walking}

\section{Adaptive Ball Walking}

\section{Hit and Run}

\section{Implementation Details}

\section{Results}

\section{Conclusion}

\pagebreak

\begin{thebibliography}{99}

\bibitem{Vempala} L. Lov\'asz and S. Vempala. Hit-and-Run from a corner. \emph{SIAM Journal on Computing}, 35(4):9851005, 2006.

\bibitem{Smith} R.L. Smith, \emph{Efficient Monte-Carlo procedures for generating points uniformly distributed over
bounded regions}, Oper. Res., 32 (1984), pp. 1296-1308.

\end{thebibliography}

\end{document}

